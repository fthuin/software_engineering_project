\section{Requirements}
\label{sec:Requirements}

\todo[inline]{Relire et/ou corriger}

In this second section, we will present the various requirements on which we
based the development of the website. All the requirements were given by the
client in paper format or verbally, and we tried our best to fulfill most of
them. Because of the amount number, we will only develop the more interesting
and more significant ones.

\subsection{Players registration}
\label{sub:Players registration}

\subsubsection{Player registration form}
\label{subs:Registration form}

\todo[inline]{Relire et/ou corriger}

The first requirement is to allow each participant of the tournaments to
register on the website. This is done through the user registration form,
accessible from the front page of the website in the top-right menu.
In this form, the participant has to fill in all the information needed to
create a new account, such as his or her personal email address or the account
password, but also information about him or herself (at least the ones that
are mandatory in order to be enlisted into a tournament).
%TODO
\todo[inline]{pas très clair, à modifier ou retirer : "Some of these fields
are limited by the administrator in charge of the registration process,
such as the gender or the player's ranking. Each field is then verified to be
sure that everything has the good format and nothing is missing."}
\bigskip

Once everything is checked, an email is sent to the given email address,
including a link to confirm the email address, indispensable to complete
the registration process. It is also important to notice that neither the
enrollment to a tournament nor the registration of a court will be accessible
until the email address has been confirmed. All the personal information can
be edited later on by the user, through the page of his or her profile edition,
on the top-right menu when logged in. Finally, it is during the registration
process that the geolocalisation is computed, based on the physical address
of the player, allowing us to integrate the eco-friendly requirement requested
by the client. \newline

We have chosen to work with accounts instead of the old registration process.
The advantage of a account-wide access is that everything becomes way easier
to manage the data of the players, and which consequently eases many other
processes such as the management of pairs, groups, and tournaments as a whole.

\subsubsection{Player management}
\label{subs:Player management}

\todo[inline]{Relire et/ou corriger}

After the player's registration, it is necessary to be able to manage all the
registered players on the website. All of it is made possible through the staff
pages of the website, in the user management tab, which is only accessible
to authorized staff members. In this tab, all the registered users on the
website are available for consultation with a single click in a list of users.
The information of a user can be modified if needed by the staff members
themselves. It is also possible to perform a search on one or several fields,
such as the name, address, or the payment status, in order to slim the number
users in the list.

\subsection{Pair registration}
\label{sub:Pair registration}

\subsubsection{Pair registration form}
\label{subs:Pair registration form}

\todo[inline]{Relire et/ou corriger}

Once a player is registered on the website, he has to find a partner in order
to compete in any available tournament. Since each player has only one account
for him or herself, he or she should be able to constitute a pair with another
player of the website. This is done through the pair registration form,
which only accessible for users with a confirmed email address. In this form,
the user can go through all the others players in the system, and select
the one he wants to pair with. To make this process easier, the user also has
the possibility to perform a search on several fields, such as the first or
last name, to find his or her partner. \newline

Once the partner has been selected, the user can select between several extra
services. He also has the opportunity to leave a comment, for wishes and/or
remarks, in a text area at the bottom of the page. When the user sends the
pair registration demand to the partner, several automated checks are performed
to assign the pairs to the correct tournament. These checks are based
on the criteria given by the client for each tournament, that is the age and
gender of the players. \newline

Once the pair is created and correctly assigned to a tournament, a mail
is sent to each player of the pair with a summary of all the information on the
newly created pair.

\subsubsection{Payment methods}
\label{subs:Payment methods}

\todo[inline]{Relire et/ou corriger}

The payment methods are available to a pair after the enlistment to one of the
tournaments. Each member of the pair can chose to pay for the inscription and
extra fees, with several payment methods (MasterCard, Visa, Paypal or credit
transfer). After the payment of the fees, the status of the pair is updated,
with the field \textit{Pair payment validated by the staff} ticked.\newline

The payment methods page is implemented, but the methods themselves aren't.
This is explained by the fact that these methods required the actual accounts
and informations of the client to be fully functional, which we were of course
not given. We thus decided not to implement the payment process with false
accounts, which would have been replaced anyway by the client. At the end,
everything is in place for the payment to be operational, exception made for
the setup of the client's accounts.

\subsubsection{Extra management}
\label{subs:Extra management}

\todo[inline]{Relire et/ou corriger}

Another requirement was the ability to propose multiple extra services to the
players. This is possible in the \textit{Extra} tab, where news extras can be
created and existing ones modified. Their names, description and price are
displayed and editable, and all the players that ordered these extras are
visible on the screen. The date of the next tournament and the inscription fees
are also modifiable in this tab.

\subsubsection{Pairs management}
\label{subs:Pairs management}

\todo[inline]{Relire et/ou corriger}

The same way the player management tab allows the staff to interact with all
the players registered on the website, the pair management tab is the page
where all the pairs are accessible to the staff. These pairs are displayed in a
very user-friendly manner, with all the informations of both players in the
pair obtainable and editable in one click. The pairs informations are also
editable if needed. \newline

A search function is also present, with the same mechanics as the one in the
player management tab (one or multiple criteria, with the results being
displayed directly on the page). Each staff member has also the possibility to
print a pdf version of a given pair, by clicking on the printer icon.

\subsection{Courts registration}
\label{sub:Courts registration}

\subsubsection{Court registration form}
\label{subs:Court registration form}

\todo[inline]{Relire et/ou corriger}

As soon as a player is registered on the website, he has the possibility to
register one or multiple tennis courts that will be used for the tournaments.
This can be done through the court registration form, directly accessible from
the home page when the player is identified on the website. Several fields are
needed (some of them can have limited choices imposed by the staff), such as
the exact address and location of the court or the availabilities of the court.
Once every mandatory field is filled, several automated checks are performed to
be sure that the informations are valid. The address of the court is also
checked by geo-location using Google Map. Once every field is checked and
validated, the new court is created and displayed under the court tab of the
website.

\subsection{Group creation}
\label{sub:Group creation}

\todo[inline]{Relire et/ou corriger}

One of the biggest and more challenging requirement of this project was the
group creation. Each tournament had to be composed of several different groups,
where players were dispatched according to their rank, age, or even their
ecological footprint (meaning the distance between their house and the court
they have to play on) using the geo-location. This is achieved on the
tournament tab, available to authorized staff members. On this page, all the
groups for each tournament can be automatically generated according to the
previous criteria. The number of groups and the number of players per groups
can also be adjusted using the corresponding fields. Once everything is set,
one click on the generate button creates the groups. \newline

After the groups are generated, it is still possible to modify them, by
swapping two players of different groups. This can be done by drag and dropping
the desired player onto the other one, or by clicking on the first player then
on the second one. Each group also has to have a leader and a court to play on.
This is made possible by clicking on the corresponding buttons that
automatically assign one, or they can be picked by hand by using the
corresponding fields. The non-automated method can be used to reduce even more
the ecological footprint, by picking a leader living the closest to the
headquarters, or by assigning the courts the closest to the players houses.
Finally, all the requests and comments made by the players and the court owners
are displayed, leaving indications for the staff. \newline

Once everything is set, the groups can be either saved (and modified later on)
or registered, meaning that they can no longer be adjusted. A mail is then sent
when a group is validated by an authorized member of the staff.

\subsubsection{Group scores}
\label{subs:Group scores}

\todo[inline]{Relire et/ou corriger}

Once the groups are created and the first matches played, the players can enter
and submit their score onto the tournament scoreboard available on the website.
All these score are not definitive, an authorized member of the staff has to
validate them to be sure that nothing is  wrong or missing. If none of the
players submit their scores, the staff members can enter manually all the
results onto the website, based on the printed scoreboard that each player
returns after their matches. The staff members can edit and save all the scores
before validating them for good, allowing them to gradually completing the
board as the scores comes. Once all the score are entered and the scoreboard is
validated, the points are attributed to the winners and the afternoon
tournament can be generated. \newline

We thought that allowing the players to enter their scores on the website was a
good idea, mainly to reduce the work of the staff members. There is no risks of
fraud, because at the end it is always a staff member that verifies and
validates each scoreboard.

\subsubsection{Printable version of groups}
\label{subs:Printable version of groups}

\todo[inline]{Relire et/ou corriger}

An important requirement was to be able to print the groups of a given
tournament, with all the necessary informations (the empty scoreboard, the
players, the courts, the sponsors, and so on). On the group tab, we allow the
staff members to print a pdf version of a given group or all groups of a
tournament, with these informations.

\subsection{Knock-off tournaments}
\label{sub:Knock-off tournaments}

\subsubsection{Knock-off tournament creation}
\label{subs:Knock-off tournament creation}

\todo[inline]{Relire et/ou corriger}

After the completion of the morning scoreboard, the afternoon matches have to
be determined. The authorized staff members can choose which groups will play
in the afternoon by ticking the corresponding boxes on the score tab of the
tournament. Once all the players have been picked, the staff members can build
by hand (by drag and dropping the players) the knock-off tree and decide who's
going to face whom. If there is an even number of player, each pair will have
an opponent and the tree will be symmetrical. On the other hand, if there is an
odd number of players, the staff has to decide one pair that will directly pass
the first round (based on their ranking, or any other criterion).\newline

After every new knock-off tournament created, an automated mail system will
send messages to the players with their informations (their opponents, the
court they will be playing on and so on).

\subsubsection{Knock-off tournament management}
\label{subs:Knock-off tournament management}

\todo[inline]{Relire et/ou modifier}

Another requirement was the management of these knock-off tournaments. First of
all, each knock-off tree is printable, in addition to the court details. These
printings are mostly helpful to the players, as they can get any practical
informations out of them (where, when they have to play and against whom). \newline

Once the first round of matches were played, all the scores can be directly
entered onto the tree branches (since they are clickable, the staff members
only need to select them to access the score fields). Based on the scores of
the first round, the winners are selected and the tree automatically fills up
his branches with their names. The next matches are thus immediately visible;
this process goes on until every match is played and the winner is selected.

\subsection{Staff management}
\label{sub:Staff management}

\subsubsection{History management}
\label{subs:History management}

\todo[inline]{Relire et/ou modifier}

When several members of the staff are working on the same part of the website,
it can be useful to know who modified a given value, or what has been modified.
This was implemented on the history tab, accessible by the staff members. On
this page, the date, name of the staff member, section of the website, data
modified and details are displayed.

\subsubsection{Permissions management}
\label{subs:Permissions management}

\todo[inline]{Relire et/ou Modifier}

On such an extensive project with various position (managing the courts, the
players, the pairs, creating the groups, and so on), assigning each member of
the staff to a precise job can be a convenient way to work with. We choose to
make this possible, by allowing the administrator to assign by hand which part
of the website is attributed to which member of the staff. On the permission
tab, every member of the staff is displayed and can be selected. Once a staff
member is picked, the intended permissions can be ticked on the right panel,
giving him the authorization to operate on certain parts of the website.
