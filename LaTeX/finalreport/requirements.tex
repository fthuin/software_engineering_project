\section{Requirements}
\label{sec:Requirements}

\todo[inline]{Faire une introduction qui dit que tous ces requirements sont
ceux du client, qu'ils ont été la base de notre travail, etc.}

\subsection{Players registration}
\label{sub:Players registration}

\subsubsection{Player registration form}
\label{subs:Registration form}

\todo[inline]{Dire que ça crée un compte. Dire que les champs sont vérifiés,
qu'il reçoit un email de confirmation et que tant qu'il a pas cliqué sur le
lien, il n'a pas accès à l'inscription ou à l'enregistrement d'un terrain.
Dire que la géolocalisation se calcule ici. Dire que certains champs sont
limités par l'administrateur (sexe, classement,\ldots). Dire que tous ces
champs sont modifiables plus tard par l'utilisateur.}

\subsubsection{Player management}
\label{subs:Player management}

\todo[inline]{Dire qu'il y a une interface staff réservée à la gestion des
utilisateurs dans la partie staff qui permet au staff de gérer les informations
des joueurs, de faire des recherches selon différents critères, etc.}

\subsection{Pair registration}
\label{sub:Pair registration}

\subsubsection{Pair registration form}
\label{subs:Pair registration form}

\todo[inline]{Dire que la personne peut rechercher les autres joueurs dans
une liste ou à travers une fonction de recherche, il peut choisir les extras
proposés par le staff. Des checks automatisés sont faits pour assigner les
joueurs à des tournois en fonction des critères donnés par le client. Dire
qu'un mail récapitulatif est envoyé lorsqu'une paire est validée.}

\subsubsection{Payment methods}
\label{subs:Payment methods}

\todo[inline]{Dire qu'une page pour les méthodes de paiement est déjà
implémentée mais que les méthodes de paiement ne le sont pas car de toute
façon elles devraient être mises à jour par le client; ça n'avait donc pas
de sens de faire des faux comptes si le client doit tout réinstaller derrière}

\subsubsection{Extra management}
\label{subs:Extra management}

\todo[inline]{Dire que le staff est capable de modifier les frais d'inscription
de modifier la date du prochaine tournoi, de modifier les extras, de voir le
nombre de joueurs qui ont demandé un extra, etc}

\subsubsection{Pairs management}
\label{subs:Pairs management}

\todo[inline]{Dire que le staff a une partie réservée à la gestion des paires
avec des fonctions de recherche par critères et une affichage simple qui leur
permet d'accéder à des vues d'affichage ou de modification d'informations
sur les paires ; ainsi que la possibilité d'imprimer une paire.}

\subsection{Courts registration}
\label{sub:Courts registration}

\subsubsection{Court registration form}
\label{subs:Court registration form}

\todo[inline]{Dire que tous les champs sont vérifiés, que l'adresse est
vérifiée pour géolocalisation par Google, que certains champs sont limités par
défaut par l'administrateur (état du terrain, type, surface, \ldots)}

\subsection{Group creation}
\label{sub:Group creation}

\todo[inline]{Expliquer la gestion des poules, dire qu'on peut faire du
drag\& drop, du click\& swap, qu'on peut générer automatiquement toutes les
poules grâce à la géolocalisation pour diminuer l'empreinte écologique. Dire
que si le staff modifie la génération automatique, il peut quand même diminuer
l'empreinte écologique en assignant les terrains les plus proches des joueurs
ou en assignant le joueur le plus proche du quartier général en tant que group
leader. On leur affiche les commentaires sur le terrain et les commentaires des
joueurs sur cette page. Dire qu'un mail est envoyé lorsqu'un groupe est validé
par un staff.}

\subsubsection{Group scores}
\label{subs:Group scores}

\todo[inline]{Dire que les utilisateurs peuvent proposer des scores que le
staff peut voir lorsqu'ils enregistrent les tableaux de scores. Les staffs
entrent manuellement les scores dans le tableau, ils peuvent faire des
sauvegardes. Une fois que tous les scores seront entrés, ils pourront créer
le tournoi de l'après-midi.}

\subsubsection{Printable version of groups}
\label{subs:Printable version of groups}

\todo[inline]{Dire que le site permet d'imprimer les poules unes par une ou
d'imprimer toutes les poules relatives à un tournoi, que ça comprend à la fois
un tableau de scores, les informations sur le staff, les informations sur les
joueurs, les terrains, les sponsors,\ldots}

\subsection{Knock-off tournaments}
\label{sub:Knock-off tournaments}

\subsubsection{Knock-off tournament creation}
\label{subs:Knock-off tournament creation}

\todo[inline]{Dire qu'après l'insertion des scores pour le matin, le staff
peut choisir les poules qui joueront l'après-midi en cochant des cases. Une
fois les personnes cochées, on propose au staff un drag\& drop pour créer
l'arbre de l'après-midi et choisir qui joue contre qui ; et si un nombre
impair de paires a été choisi, qui passera les étapes sans jouer de matchs.
Dire que des mails automatisés sont envoyés lorsqu'un tournoi est créé.}

\subsubsection{Knock-off tournament management}
\label{subs:Knock-off tournament management}

\todo[inline]{Dire qu'il est possible d'imprimer l'arbre de tournoi et les
informations du terrain pour les joueurs. L'arbre a toutes ses branches qui
sont cliquables pour insérer les scores et en fonction des scores l'arbre
se remplit automatiquement avec les prochains matchs à jouer.}

\subsection{Staff management}
\label{sub:Staff management}

\subsubsection{History management}
\label{subs:History management}

\todo[inline]{Dire que les actions des staffs sont affichées sur une page
spécifique de l'interface staff et elle permet de voir qui a modifié quoi}

\subsubsection{Permissions management}
\label{subs:Permissions management}

\todo[inline]{Expliquer qu'on peut assigner des permissions à chaque utilisateur
qui a créé un compte sur le site pour lui permettre d'accéder à certaines
parties de l'interface staff.}
