\section{Architecture choices}
\label{sec:Architecture choices}

In this first section, we will present the numerous architectural choices that
we have made while developing our web application. In order to lighten the
explanation, we have separated those choices in two categories. The
\textit{back-end} category gathers all the tools that are related
to the web server and the database, and the \textit{front-end} category gathers
all the tools that are handling the user experience of the website.

\subsection{Back-End architecture}

We have chosen to use take advantage of the framework \textbf{Django}, which is
an open-source (and thus, free-to-use) framework which is renowned for its
security, its complete administrator management tools, its ease to manage a
database, and finally its fairly large support community. This framework is quite
recent in comparison to the less secured PHP frameworks, while being mature and
not at the bleeding edge of new technology where errors support is not
guaranteed. All of it makes \textit{Django} an excellent candidate overall, and
this technology ensures that it will not become outdated before a very long
time.\newline

This framework uses Python as the programming language, which happens to be
very easy to learn in case the software needs to be modified. \textit{Django}
is relatively accessible so any programmer should be able to make any desired
modification with the help of our developer guide documentation, which explains
how to understand, use, and improve our code.\newline

As for the database, we have chosen to use PostgreSQL, which is an open-source
database that is well recognized for its data reliability. If you wish to
change to an SQLite or MySQL database, for whatever reason, it would require
next to no effort in the code to make the change, as \textit{Django}'s database
management writes database queries by itself. So you just have to specify the
type of database that you are using, and \textit{Django} will handle the
queries by itself. \textit{Django} does also the migration when you change tables
properties which is very nice for a big project in which you can't predict
everything. \newline

\subsection{Front-End architecture}

We have chosen to use a lot of tools to make the design of the website. The
three main tools are HTML (which let us create website elements), CSS (which
let us customize the elements with color, placement, and more) and JavaScript
(which let us build sophisticated behavior, such as validation, animation,
forms, etc\ldots{}). Similarly to the back-end side, there are some frameworks
that are built around those tools, making the work on the design easier, better,
and faster. \newline

One of these tools is \textbf{Bootstrap}, a framework that is using HTML, CSS
and JavaScript for developing responsive website for desktop and mobile
monitors. We added other JavaScript libraries such as \textbf{jQuery} or
\textbf{Gracket} to add some nice features the website needed. \newline

With all those frameworks and some work, we were able to develop a design that
can easily be modified by any competent programmer, should the need arise.
