\section{Teamwork}

\subsection{Romain}
The overall teamwork of the group is really good: the workload is nicely
split between all the members for each sprint; the weekly meetings allow
us to evaluate the progress of the project; the group leader assures that
everything goes according to the initial planning, and everybody is ready to
help if something takes more time than expected. \newline

\subsection{Benjamin}

This project lets us work as a small team, in conditions that are closer to the working world than any other academic project we had. The group has to be very careful on the planning, but our team is capable of efficiently splitting the work among the members until now. The work is accomplished, and only external things, such as a bunch of unexpected work from other courses, have a bad impact on our work flow for this project. \newline

Furthermore, the atmosphere of the group gives motivation to all of us: everyone stays ready to help anyone, avoiding the group to stuck indefinitely on a bug.

\subsection{Quentin}
We had a slow start for the first week of implementation. This is due to some peoples of the group having to get familiar with the farmeworks. Beside that everybody on the group is willing to create a good website that match the client's expectation.\newline

This is the largest project we have ever had and a great challenge to work with a larger group of people. This allows us to get more familiar with bigger projects that we will have to face after university. 


\subsection{Zacharie}
\subsection{Alexandre}

For some of us, it is the first time we work together on a project. It has provided a great advantage to the group, because we have quickly detected the abilities of everyone. \newline

As a result, the tasks are fairly divided up, and the work is progressing at an appropriate rate, all while having a very good atmosphere within the group.

\subsection{Florian}
It is a significant amount of work to create a website in accordance to the customer requirements. It is hard to have a global view on the project, because the requirements hide many untold technical things on the desired product. This can lead to some difficulties with regard to the workload evaluation. Moreover, it is a challenge to manage a team of eight people, especially as it is a first time for me. Fortunately, my
teammates are all qualified to help in (at least) one part of the workload for
each sprint, and everyone is pretty self-governing. \newline

The main difficulty so far is to push everyone to deliver his work part by
part (beginning by the structure and then producing a step-by-step
implementation; each time with a commit on the git) to ensure the progress
of the tasks and to help when other team members have free time.
Personally, I feel that everyone does what he can, despite the workload in
the first year of master which limits the time left for this project.\newline
\subsection{Cyril}

I have no problem with the team in general, everything is proceeding
smoothly along the way as everyone accomplish the task they have been given
flawlessly. In short, everything is close to perfect in the team in my
opinion and I hope it stays this way until the end of the project.
\subsection{Nicolas}
It is the first time for me that i have to deal with such a big group of people to work with. The last time i had to do a big project, we were 4. Now the members are doubled. At first i though it would be incredibly difficult to work together on the same project. But at the start of the first sprint, we managed to divide the work very well. And we carried on this work dividing success through all the sprint so far. I didn't know we would be able to accomplish so much work. Even though we didn't deliver everything we wanted on the last sprint, we are still proud of what we developed and on time to deliver our final version. If we can continue on this line of good work, we will deliver a very good product to the client.
