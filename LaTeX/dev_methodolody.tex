\section{Choice of the development methodology (25/09/2015)}

\subsection{Software development method}

\subsubsection{Development method}
We decide to choose \enquote{Agile} and Scrum as a development method because
everyone in the team worked at least once with this method (in the 4th
project of computer science bachelor) and it fits perfectly with this
kind of project with 8 team members.\newline

\subsubsection{Back End development}
The back end handles everything related to the web server and the database. Using a framework will allow us to focus on the website development without worrying too much about security,use management, administration panel, etc.\newline

We are going to use \textbf{Django}, an open-source framework, for the back-end. It has a large community and is easy to learn. Some group members already have experience with this framework which is gonna make it easier for everybody to get started quickly.\newline

Another advantage of \textbf{Django} is the database management. \textbf{Django} uses Django-ORM which makes it really easy to communicate with the database. Django-ORM writes the SQL statements by itself. This allows us to use any type of database(sqllite,mySQL,PostgreSQL) and change it during the project development without impacting the rest of the project.
\textbf{Django} uses python as programming language. Python is easy to learn and we have to learn it for our Artificial Intelligence class.
Python is also very clean and easy to read thanks to it's indentation syntax. Once again, multiple team members had experience with python which is good to get started faster.\newline

We also looked at the possibility of doing our website with PHP. Beside the fact that PHP is getting old, it is still the most used back-end language on the web. It historical security problems and the fact that the code is included inside the HTML are two main reasons why we didn't chose PHP.\newline

Ruby seems like a pretty good option as well and looked a lot like python but none of our team members had experience with it. JavaScript was another option but the only framework that some team member knew with JavaScript is \textbf{NodeJS} which was used in a previous project. Most people didn't have a good experience with it(setting up the project may be long and there are a lot of differents middle wares to install to make your project running).

\subsubsection{Front-End development}

There are lots of tool to build the website design. The three main tools are HTML (creating the web site elements), CSS (customize elements with color, placement and more) and JavaScript (used for validation, animation, forms, etc). Like the back end, there are frameworks build around those tools making the design building easier, better and faster. We will mainly use \textbf{Bootstrap}. It's a framework using HTML, CSS and JavaScript for developing responsive website for desktop and mobile. \textbf{Bootstrap} alone isn't enough to build the whole website design. An other framework we are planning to use is \textbf{Angular JS}. It extends HTML with new attributes to handle events, forms, input, validation and more. With those two framework we'll be able to build to most part of the application. We'll just need to add some more CSS to customize it.

\subsubsection{Management}

We are using \textbf{Git} to have a repository for our codes and reports or the project. We'll also use \textbf{Trello}, a free web-based project management application using kanban paradigm. Both \textbf{Git} and \textbf{Trello} were used in the project 4 last year and we all are familiar with these tools.

We will use Trello as a kanban and GitHub as a host for the code. \newline

\subsection{Initial planning}



\subsection{Report on requirement analysis}

\subsubsection{Players registration}

\paragraph{Player data}

We have to create a web based form. Check for previous data to avoid
redundancy. \newline

\begin{itemize}
    \item Gender
    \item Last name
    \item First name
    \item Postal address + number + post box + postal code
    \item Phone number + Mobile phone number
    \item Date of birth
    \item Ranking (AFT)
    \item Email address
    \item First time participation or not
\end{itemize}

\paragraph{Check AFT ranking}
\paragraph{Player location}
\paragraph{Options}

Player can ask for meals and add comments when they register for the
tournament.

\paragraph{Mailing}

\subsubsection{Courts registration}

\paragraph{Court data}

We have to create a web based form. Check for previous data to avoid
redundancy.

The data should minimum contains :

\begin{itemize}
    \item (ID)
    \item Court number
    \item Address of the court
    \item Court surface (multiple choice)
    \item Type of court (private owner, club - multiple choice)
    \item Geographical zone
    \item (maybe we can find an alternative)From now, on which map it is
        located, a large plan is given to the players in the afternoon
        to find the courts.
    \item Special instructions for accessing the court
    \item Address of the owner
\end{itemize}

\paragraph{Preferences}
\paragraph{Mailing}

\subsubsection{Group creation}

\paragraph{Linking two players}
\paragraph{Mailing}

\subsubsection{Knock-off tournament}
\subsubsection{Staff user accounts}

\paragraph{Dating the modification}
\paragraph{Editing every information}
\paragraph{Automated tournament steps following results}
\paragraph{Personal Documents}
\paragraph{Access to a forum}

\subsection{Use cases}
\subsubsection{Registration of a player}
\subsubsection{Registration of the owner of a court}

As a court owner, I can register my court with a bunch of information :

\begin{itemize}
    \item (ID)
    \item Court number
    \item Address of the court
    \item Court surface (multiple choice)
    \item Type of court (private owner, club - multiple choice)
    \item Geographical zone
    \item (maybe we can find an alternative)From now, on which map it is
        located, a large plan is given to the players in the afternoon
        to find the courts.
    \item Special instructions for accessing the court
    \item Address of the owner
\end{itemize}



\subsubsection{Creation of a group}
