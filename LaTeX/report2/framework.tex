%Frameworks
\section{Framework}

After having made our decision upon the software architecture, we must now
choose which framework to use as to respect this architecture and ease the
implementation of the website. In the three sections below, we detail the
different frameworks that will be used to implement the various parts of
the website and motive our choice of those specific frameworks. \newline

Of course, all frameworks that we even considered can be used to implement
a 3-tier client-server architecture as they would be useless for us
otherwise.

\subsection{Back-End}

The back-end handles everything related to the web server and the database.
Using a framework will allow us to focus on the website development,
without worrying too much about security, user management, administration
panel, etc.\newline

After a long process of careful consideration and investigation, we decided
on using \textbf{Django}, an open-source framework, for the back-end. It
has a large community and is easy to learn. As a bonus, some group members
already have some experience with this framework, making it easier for
everybody to get started quickly.\newline

The main advantage of this framework being his architecture which is
extremely well suited for implementing system based on a 3-tier
client-server architecture (See figure 3). As you can see, \textbf{Django}
relies on multiple layer of processes to handle request as swiftly and
efficiently as possible. Each process having it's own role and being called
only if necessary.  \newline

\begin{enumerate}
\begin{figure}
    \begin{minipage}{\linewidth}
        \centering
        \begin{minipage}{0.45\linewidth}
                \item The URL dispatcher (urls.py) maps the requested URL
                    to a view function and calls it. If caching is enabled,
                    the view function can check to see if a cached version
                    of the page exists and bypass all further steps,
                    returning the cached version, instead. Note that this
                    page-level caching is only one available caching option
                    in Django. You can cache more granularly, as well.
                \item The view function (usually in views.py) performs the
                    requested action, which typically involves reading or
                    writing to the database. It may include other tasks, as
                    well.
                \item The model (usually in models.py) defines the data in
                    Python and interacts with it. Although typically
                    contained in a relational database (MySQL, PostgreSQL,
                    SQLite, etc.) other data storage mechanisms are
                    possible as well (XML, text files, LDAP, etc.)
                \item After performing any requested tasks, the view
                    returns an HTTP response object (usually after passing
                    the data through a template) to the web browser.
                    Optionally, the view can save a version of the HTTP
                    response object in the caching system for a specified
                    length of time.
        \end{minipage}
        \begin{minipage}{0.05\linewidth}
            \vline
        \end{minipage}
        \begin{minipage}{0.45\linewidth}
            \begin{center}
            \includegraphics[width=\linewidth]{DjangoArchitecture.png}
            \caption{Diagram of Django's architecture}
            \label{fig:length_eight_mouse}
            \end{center}
                \item Templates typically return HTML pages. The Django
                    template language offers HTML authors a simple-to-learn
                    syntax while providing all the power needed for
                    presentation logic.
                \item Templates typically return HTML pages. The Django
                    template language offers HTML authors a simple-to-learn
                    syntax while providing all the power needed for
                    presentation logic.
        \end{minipage}
    \end{minipage}
\end{figure}
\end{enumerate}

Another advantage of \textbf{Django} is the database management.
\textbf{Django} uses Django-ORM which makes it really easy to communicate
with the database. Django-ORM writes the SQL statements by itself. This
allows us to use any type of database (sqLite, mySQL, PostgreSQL) and
change it during the project development without impacting the rest of the
project. \textbf{Django} uses python as programming language. Python is
easy to learn and, as we have to use it for our Artificial Intelligence
course, learning it won't increase the work load at all.
Python is also very clean and easy to read thanks to its indentation
syntax. Once again, multiple team members had experience with python, which
is best to get started as quickly as possible.\newline

We also looked at the possibility of doing our website with \textbf{PHP}.
Beside the fact that \textbf{PHP} is getting very old, it is still the most
used back-end language on the web. But, its historical security problems
and the fact that the code is included inside the HTML, are two important
reasons that deterred us from choosing \textbf{PHP} for this
project.\newline

\textbf{Ruby on rail} also seemed like a pretty good option and looked a
lot like python, but none of our team members had experience with it,
neither with the ruby language in general. Using Javascript was another
option and some of our team member already used \textbf{NodeJS} in a
previous project. But most of the people who had experience with this
framework deterred us from using it again as they had a pretty bad
experience with it (setting up the project may be long, there are a lot of
different middle-wares to install to make your project run, community not
always so helpful, ...).

\subsection{Front-End}

There are lots of tools to build a website design. The three main tools
being HTML (creates the website elements), CSS (customizes elements with
color, placement, and more) and JavaScript (used for validation, animation,
forms, etc). Like for the back-end, there are frameworks built around those
tools, making the work on the design easier, better, and faster. We will
mainly use \textbf{Bootstrap}. It is a framework using HTML, CSS and
JavaScript for developing responsive website for desktop and mobile.
\textbf{Bootstrap} alone is not enough to build the whole website design.
Another framework we are planning to use is \textbf{Angular JS}. It extends
HTML with new attributes to handle events, forms, inputs, validations, and
more. Allowing our website to be more efficient in general. \newline

With those two frameworks, we will be able to build most parts of the
application. We might just need to add some more CSS to customize it if the
need arise.

\subsection{Management}

We are using \textbf{Git} to have a private repository for our codes and
reports of the project. We will also use \textbf{Trello}, a free web-based
project management application using the Kanban paradigm. Both \textbf{Git}
and \textbf{Trello} were used in the programming project of last year, so
we all are quite familiar with these tools. Finally to maintain and follow a clear planning we will make use of \textbf{Wrike}, an organisational online tool that is free to use for small team.\newline

We will use Trello as a Kanban, and GitHub as a host for the code. \newline