\chapter{Ajouter une page au site}

\section{Création du Front-End}

Avant de vouloir ajouter de nouvelles pages au site web, il est nécessaire de savoir que le framework Django utilisent des templates. Ces templates sont écrites en langage HTML mais permettent l'intégration de certaines expressions et structures comme les \textit{if...else...} ou les \textit{for}. Ces expressions et structures doivent être entourées par des balises \textit{\{\% ... \%\}} comme on peut le voir sur la figure~\ref{fig:Utilisation d'une expression if...else... dans une template}\\

\begin{figure}[!ht]
\centering
\begin{framed}
\lstinputlisting[language=HTML, firstline=1, lastline=8]{"developer_guide/template.html"}
\end{framed}
\caption{Utilisation d'une expression \textit{if...else...} dans une template}
\label{fig:Utilisation d'une expression if...else... dans une template}
\end{figure}
\FloatBarrier

Afin d'inclure de nouvelles fonctionnalités, il est possible d'ajouter de nouvelles pages au site. Pour ce faire, il faut dans un premier temps créer un fichier HTML et le placer dans le dossier \textit{ASMAE/tennis/templates}. Il est intéressant de noter que les différents éléments statiques (style, polices, images ou scripts) doivent être ajoutés dans les dossiers correspondants (css, fonts, img, js)  présents dans le répertoire \textit{ASMAE/tennis/static/tennis}.\\

Pour obtenir un design de base correspondant à l'ensemble du site web (couleurs, en-têtes, arrière plan, etc.), il est nécessaire d'ajouter au-début du fichier HTML la ligne suivante:

\begin{verbatim}

\end{verbatim}

Cette ligne va permettre d'inclure et d'étendre le fichier \textit{base.html} qui contient le squelette des pages du site. Le tag \textit{\{\% block \%\}} présent à plusieurs reprises dans ce fichier nous permet de redéfinir certaines parties concernées par ce tag dans notre nouvelle template. Le tag \textit{\{\% urls \%\}}, quant à lui, va nous permettre de créer des liens vers d'autres vues. D'autres tags existent et sont expliqués dans la documentation Django disponible à cette adresse: \url{https://docs.djangoproject.com/fr/1.9/ref/templates/builtins/}

\section{Gestion du Back-End}

\subsection{Déclarer la vue et lui associer une URL}

Pour représenter une vue, Django utilise une fonction définie dans le fichier \textit{views.py} (se trouvant dans le dossier \textit{ASMAE/tennis}). Cette fonction va être utile pour récupérer certaines données (de la base de données par exemple) et générer le rendu du fichier HTML créé précédemment. Un exemple d'une fonction définie dans le fichier \textit{views.py} est repris sur la figure~\ref{fig:Exemple d'une fonction présente dans le fichier views.py}. 

\begin{figure}[!ht]
\centering
\begin{framed}
\lstinputlisting[language=Python, firstline=1, lastline=2]{"developer_guide/view.py"}
\end{framed}
\caption{Exemple d'une fonction présente dans fichier \textit{views.py}}
\label{fig:Exemple d'une fonction présente dans le fichier views.py}
\end{figure}
\FloatBarrier

Maintenant que la vue est déclarée, il ne reste plus qu'à l'associer à une URL. Pour ce faire, il faut modifier le fichier \textit{urls.py} (également présent dans le dossier \textit{ASMAE/tennis}) qui contient toutes les associations entre les liens URL disponibles sur le site et les vues. Il suffit alors de rajouter une nouvelle ligne dans le fichier, plus précisément dans \textit{l'urlpatterns}, en respectant le format suivant : \textit{url(NouveauLienUrl, NomDeLaViews)}. Nous pouvons observer un exemple de fichier \textit{urls.py} sur la figure~\ref{fig:Exemple d'un fichier urls.py}. A noter que le \textit{r'\^\$'} équivaut à spécifier la racine du site web.

\begin{figure}[!ht]
\centering
\begin{framed}
\lstinputlisting[language=Python, firstline=4, lastline=19]{"developer_guide/view.py"}
\end{framed}
\caption{Exemple d'un fichier \textit{urls.py}}
\label{fig:Exemple d'un fichier urls.py}
\end{figure}
\FloatBarrier

%Ajouter passage d'arguments aux vues

La nouvelle page est désormais intégrée au site web et est disponible à l'adresse spécifiée dans le fichier \textit{urls.py}

\subsection{Récupération des informations de la base de données vers les pages web}

Pour transmettre des informations vers les différentes pages web, il va falloir utiliser les vues et par conséquent modifier le fichier \textit{views.py}. Les vues vont être utilisées pour récupérer des informations de la base de données et les envoyer vers la page web.

\begin{figure}[!ht]
\centering
\begin{framed}
\lstinputlisting[language=Python, firstline=20, lastline=24]{"developer_guide/view.py"}
\end{framed}
\caption{Vue utilisée pour la gestion des terrains par les utilisateurs}
\label{fig:Vue utilisée pour la gestion des terrains par les utilisateurs}
\end{figure}
\FloatBarrier

Sur la figure~\ref{fig:Vue utilisée pour la gestion des terrains par les utilisateurs}, nous testons via la requête passée en argument si l'utilisateur courant est connecté. Si tel est le cas, nous allons alors récupérer l'ensemble des terrains enregistrés par l'utilisateur courant dans la base de données via la méthode \textit{objects.filter(user=request.user)} appliquée sur la table \textit{Court}. Une fois les terrains enregistrés dans une variable, nous pouvons envoyer cette dernière vers la page web grâce à la méthode \textit{render()} qui prend comme arguments la requête, le nom de la template et un contexte d'instances. Ce dernier peut être obtenu avec la méthode \textit{locals()} qui va retourner un dictionnaire contenant toutes les variables locales de la fonction d'où elle a été appelée (et donc notamment l'ensemble des terrains enregistré dans la variable \textit{court}).\\

Maintenant que nous avons envoyé les informations souhaitées vers la page web, nous pouvons dès à présent les récupérer dans le fichier template correspondant d'une façon telle que celle présentée sur la figure~\ref{fig:Récupération des informations envoyées vers la template}. Sur cette dernière, nous constatons que nous pouvons directement interagir sur la variable \textit{court} et afficher les informations souhaitées grâce aux balises \textit{\{\{ ... \}\}}.

\begin{figure}[!ht]
\centering
\begin{framed}
\lstinputlisting[language=HTML, firstline=10, lastline=17]{"developer_guide/template.html"}
\end{framed}
\caption{Récupération des informations envoyées vers la template}
\label{fig:Récupération des informations envoyées vers la template}
\end{figure}
\FloatBarrier

Il est important de préciser que d'autres requêtes que celle utilisée dans la figure~\ref{fig:Vue utilisée pour la gestion des terrains par les utilisateurs} peuvent être appliquées. La liste de ces requêtes est reprise sur la documentation Django disponible à l'adresse suivante:\url{https://docs.djangoproject.com/en/1.9/topics/db/queries/}

\subsection{Enregistrer des informations des pages web vers la base de données}

Pour récupérer des informations d'une page web et les enregistrer dans la base de données, il va également être nécessaire d'utiliser les vues. Cependant, nous devons d'abord établir le lien entre les templates et le fichier \textit{views.py}. Pour ce faire nous pouvons utiliser les formulaires. Ces derniers constituent le moyen le plus simple pour enregistrer des données entrées par l'utilisateur dans la base de données. Un exemple de formulaire est décrit dans le fichier \textit{register.html}.\\

Une fois le formulaire correctement défini dans la template, nous pouvons nous rendre dans le fichier \textit{views.py} et modifier la fonction définissant la vue correspondante à la template. La figure~\ref{fig:Récupération des informations envoyées par l'utilisateur} décrit la récupération des données et l'enregistrement d'un nouvel utilisateur dans la base de données. 

\begin{figure}[!ht]
\centering
\begin{framed}
\lstinputlisting[language=Python, firstline=26, lastline=40]{"developer_guide/view.py"}
\end{framed}
\caption{Récupération des informations envoyées par l'utilisateur}
\label{fig:Récupération des informations envoyées par l'utilisateur}
\end{figure}
\FloatBarrier