\chapter{Hiérarchie des dossiers de contenus}

Tout d'abord, il est important de savoir qu'un projet Django se sépare en
plusieurs parties. Une partie est destiné à l'entièreté des paramètres et
des outils liés à tout le site, les autres parties sont des
\enquote{applications}. \newline

Ces \enquote{applications} ont pour but d'être modulaire, c'est-à-dire qu'elles
peuvent s'intégrer facilement dans d'autres projets Django. Dans notre cas,
il n'existe qu'une seule application : \textbf{tennis}. Elle est autonome et
gère à la fois le visuel du site et les opérations sur le serveur. \newline

Vous pouvez retrouver la hiérarchie des dossiers sur la
figure~\ref{fig:Architecture des dossiers}, chaque dossier important est
accompagnée d'une explication du contenu que vous y trouverez.

\begin{figure}[!ht]
	\begin{framed}
		\dirtree{%
			.1 /.
			.2 \textbf{ASMAE} \ldots{} \begin{minipage}[t]{10cm}
								Ce répertoire contient les fichiers directement
								reliés aux paramètres généraux du projet web{.}
								\end{minipage}.
			.2 \textbf{tennis} \ldots{} \begin{minipage}[t]{10cm}
								Ce répertoire contient tous les fichiers reliés
								à l'application \enquote{tennis}, c'est-à-dire
								tous les contenus reliés au Charles de
								Lorraine{.}
								\end{minipage}.
			.3 \textbf{dump\_table} \ldots{} \begin{minipage}[t]{10cm}
									Ce répertoire contient les fichiers
									permettant de sortir des {.}csv depuis
									l'interface admin (voir section \ref{sec:particularités interface administrateur}){.}
									\end{minipage}.
			.3 \textbf{management}.
			.4 \textbf{commands} \ldots{} \begin{minipage}[t]{10cm}
									Ce répertoire contient les fichiers
									permettant de créer des nouvelles commandes
									à lancer avec manage{.}py
									\end{minipage}.
			.3 \textbf{static}.
			.4 \textbf{tennis}.
			.5 \textbf{css} \ldots{} \begin{minipage}[t]{10cm}
									Ce répertoire contient les fichiers
									permettant de gérer l'affichage du site
									web{.}
									\end{minipage}.
			.5 \textbf{fonts} \ldots{} \begin{minipage}[t]{10cm}
									Ce répertoire contient les fichiers
									permettant de gérer les polices d'écriture{.}
									\end{minipage}.
			.5 \textbf{img} \ldots{} \begin{minipage}[t]{10cm}
									Ce répertoire contient les fichiers
									images utiles au site{.}
									\end{minipage}.
			.5 \textbf{js} \ldots{} \begin{minipage}[t]{10cm}
									Ce répertoire contient les fichiers
									permettant au navigateur d'effectuer des
									actions sans interaction avec le serveur{.}
									\end{minipage}.
			.6 \textbf{images}.
			.3 \textbf{templates} \ldots{} \begin{minipage}[t]{10cm}
									Ce répertoire contient les fichiers
									HTML nécessaires à l'affichage des pages
									de l'application{.}
									\end{minipage}.
			.4 \textbf{mail}.
			.3 \textbf{tests\_cases}.
			.3 \textbf{views\_helper} \ldots{} \begin{minipage}[t]{10cm}
									Ce répertoire contient les fichiers
									Python permettant de gérer les interactions
									du site avec le serveur{.}
									\end{minipage}.
		}
	\end{framed}
	\caption{Arbre décrivant l'architecture depuis le dossier racine}
	\label{fig:Architecture des dossiers}
\end{figure}
\FloatBarrier

\section{Description des fichiers importants}

\subsection{Dossier racine}

\subsubsection{manage.py}

Ce fichier permet d'interagir directement avec l'application; c'est-à-dire, lancer le site web en version locale, ajouter un superuser, effectuer les différentes migrations de la base de données, etc... Les différentes possibilités d'utilisation de l'utilitaire \textit{manage.py} sont reprises dans la documentation Django à l'adresse suivante: \url{https://docs.djangoproject.com/fr/1.9/ref/django-admin/}

\subsubsection{db.sqlite3}

Le fichier \textit{db.sqlite3} contient la base de données du site web enregistrée au format \textit{sqlite3}. Il n'est pas conseillé de modifier directement ce fichier; des outils comme l'interface administrateur ou l'utilisation de commandes dans le terminal sont disponibles pour mettre à jour le fichier.

\subsubsection{phantomjs}

Le fichier \textit{phantomjs} est crucial au bon fonctionnement de la fonctionnalité de vérification des classements des joueurs. Il s'agit en fait d'un exécutable que nous utilisons via Selenium pour simuler l'ouverture de page web et
récuperer le classement des joueurs. \newline

Les mises a jours de Selenium pourrait tôt ou tard
rendre ce fichier obsolète, une version plus récente de l'exécutable devra alors être 
téléchargé depuis le site \url{http://phantomjs.org/download.html} pour assurer le bon
fonctionnement du site web.

\subsection{Dossier ASMAE}

\subsubsection{settings.py}

Comme son nom l'indique, ce fichier permet de paramétrer entièrement l'application. Afin de bien comprendre comment ce fichier fonctionne, vous êtes invité à lire la documentation Django disponible aux adresses suivantes: \url{https://docs.djangoproject.com/en/1.8/topics/settings/} et \url{https://docs.djangoproject.com/en/1.8/ref/settings/}

\subsubsection{urls.py}

Ce fichier permet de créer de nouvelles URL pour le site. Ce premier fichier \textit{urls.py} reprend une URL pour accéder à l'interface administrateur et une URL vers le module tennis. Cette dernière URL va reprendre toutes les associations URL-views définies dans le fichier \textit{urls.py} issu du dossier \textit{tennis}.

\subsection{Dossier tennis}

\subsubsection{admin.py}

Ce fichier est nécessaire pour définir et configurer l'interface administrateur du site web. On peut donc retrouver dans ce fichier le code définissant les différentes tables de la base de données à afficher, une classe permettant d'obtenir les actions récentes de l'administrateur, etc.

\subsubsection{models.py}

Ce fichier permet d'établir la structure de la base de données et de créer les différentes tables de celle-ci. L'ajout d'une table et la configuration des champs est expliquée dans la partie \textit{ajouter une table}, chapitre \ref{ajouter une table}.

\subsubsection{mail.py}

Le fichier \textit{mail.py} est utile pour l'envoi automatique des mails. Dans ce fichier, chaque mail est représenté par une fonction. L'utilisation des mails est expliquée dans la partie \textit{gestion des mails}, chapitre \ref{gestion des mails}.

\subsubsection{classement.py} 

Ce fichier est utilisé pour vérifier le classement qu'indique chaque nouvel utilisateur (afin d'éviter les erreurs ou tentatives de triche). La vérification des classements s'effectue par le biais du site suivant: \url{http://www.classement-tennis.be/calcul.html}. Dans le cas où ce site serait à l'avenir modifé, l'outil deviendrait inefficace, mais celui-ci ne gênera pas la bonne marche des autres opérations du site. L'outil devra alors être adapté au nouveau site pour redevenir fonctionnel. Pour ce faire, une documentation de Selenium, l'outil utilisé pour réaliser cette fonctionnalité, peut être retrouvée a l'adresse suivante : \url{http://docs.seleniumhq.org/docs/}. 

\subsubsection{views.py}

Le fichier \textit{views.py} contient les méthodes nécessaires à la création des différentes views. L'utilisation de ces méthodes est reprise au chapitre \ref{ajouter une page}, \textit{ajouter une page au site}.

\subsubsection{pdfdocument.py}

Cette classe gère la création de PDF à partir de données sur le serveur en utilisant le package \textit{reportlab}, installé précédemment (voir section \ref{sec:Installation et configuration})

\subsubsection{urls.py}

Comme décrit précédemment, ce fichier permet de créer des URL en définissant des associations entre les URL et les vues (du module tennis). L'utilisation de ce fichier est également détaillée au chapitre \ref{ajouter une page}.
